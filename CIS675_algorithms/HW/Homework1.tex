
\documentclass[12pt]{article}

\usepackage{algorithm}
\usepackage{algpseudocode}

\begin{document}
\noindent
\large Homework 1:\\
 \normalsize 
 \noindent
Due February 23 at midnight Eastern time.  Submit your solutions typed and in a pdf document.  To receive full credit, explain your answers.\\

\noindent
If you collaborate with another student or use outside sources, please list those students' names and the URL/title/etc. of the sources that you referred to.  Collaboration is permitted, but you must write up your own solutions.  Be sure to fill out the cover sheet.\\

\noindent
Treat arithmetic operations (addition, subtraction, multiplication, and division) as constant time operations.\\

\noindent
1.  For all pairs of functions below, prove whether $f(n)$ is $O(g(n))$ or $g(n)$ is $O(f(n))$ (or both). \\\\
\begin{tabular}{ l l l }
       & $f(n)$ & $g(n)$ \\
  (a) & $n^3$ & $n^3 - n^2$ \\
  (b) & log$_5n$ & log $_2n$\\
  (c) & 4 log $n$ & log($n^4$)\\
  (d) & 100$n$ + (log $n)^2$ & 100$n$ + log $n$\\
  (e) & $n$! & $2^n$\\ \\
\end{tabular} 

\noindent
2.  Consider the following algorithm: 
\begin{algorithm}
\begin{algorithmic}[1]
\Function{HomeworkFunction}{array $A$}
    \State \textbf{if} $length(A) == 1$:
    \State \hspace{0.8cm}  \textbf{return} $A[0]$
    \State \textbf{else}:
    \State \hspace{0.8cm} \textbf{for} $i = 0:length(A)$
    \State \hspace{1.6cm} \textbf{print} `hi!'
    \State \hspace{0.8cm} $A1\gets A[0:length(A)/4]$
    \State \hspace{0.8cm} $A2\gets A[length(A)/4:2*length(A)/4]$
    \State \hspace{0.8cm} $A3\gets A[2*length(A)/4:3*length(A/4)]$
    \State \hspace{0.8cm} \textbf{return HomeworkFunction}(A1) $\times$ \textbf{HomeworkFunction}(A2) $\times$ \textbf{HomeworkFunction}(A3)
\EndFunction
\end{algorithmic}
\end{algorithm}

\noindent
\begin{enumerate}
\item Write the running time of this function as a recurrence relation.  Assume that the array initializations on lines 5-7 can all be done in constant time using pointers.
\item Describe the running time of this function using big-O notation.\\
\end{enumerate}
 
\noindent
3.  Suppose there are $n$ slots for scientific instruments, arranged in a row.  We have to place $n - 2$ instruments into these slots, subject to the constraint that the two empty slots cannot be next to each other.  For both parts to this problem, you must \textit{explain why} your answer is correct: giving a few examples is not enough.  (a) Write a \textit{recurrence relation} describing the number of ways to put the instruments into slots so that the empty slots are not next to each other.  (b) Write a \textit{recurrence relation} for the case where the slots are in a circle (hint: it might help if you label the slots 1, ... $n$, where 1 and $n$ are adjacent).\\

\noindent
4.  Suppose $f(n)$ is a positive-valued function defined for integers $n > 0$.  Obviously, $f(n)$ must be $O(f(n))$. Define $g(n) = f(2n)$.  Is it true that $g(n)$ must be $O(f(n))$?  If so, prove it.  If not, give an example of $f(n)$ where this does not hold.\\ 

\noindent
5.  Consider the following algorithm: 
\begin{algorithm}
\begin{algorithmic}[1]
\Function{HomeworkFunction2}{array $A$}
    \State \textbf{if} $length(A) == 1$:
    \State \hspace{0.8cm}  \textbf{return} $A[0]$
    \State \textbf{else}:
    \State \hspace{0.8cm} $A1\gets A[0:length(A)-2]$
    \State \hspace{0.8cm} $A2\gets A[1:length(A)-1]$
    \State \hspace{0.8cm} $A3\gets A[2:length(A)]$
    \State \hspace{0.8cm} \textbf{return HomeworkFunction2}(A1) $\times$ \textbf{HomeworkFunction2}(A2) $\times$ \textbf{HomeworkFunction2}(A3)
\EndFunction
\end{algorithmic}
\end{algorithm}

\noindent
\begin{enumerate}
\item Write the running time of this function as a recurrence relation.  Assume that the array initializations on lines 5-7 can all be done in constant time using pointers.
\item Determine the running time of this function using big-O notation, and prove your answer.\\
\end{enumerate}


\noindent
\textbf{Extra Credit}:  Suppose $f(n) = O(n^3)$ describes the running time of an algorithm, and you know that for input of size 10, the running time was 10 seconds; and for input of size 100, the running time was 10,000 seconds.  What can you say about the running time on input of size 500?\\

 \end{document} 

